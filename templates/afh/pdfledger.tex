\documentclass[12pt,a4paper]{report}
\usepackage{graphicx}
\usepackage{verbatim}
\usepackage{moreverb}
\usepackage{fontspec}
\usepackage[colorlinks=true]{hyperref}
\usepackage[german]{babel}
\usepackage[24hr]{datetime}
\usepackage[latin1]{inputenc}
\let\verbatiminput=\verbatimtabinput %tabs are ignored in verbatim, this corrects for that
\def\verbatimtabsize{4\relax} % set tabs=4 (else my output goes off the screen)

\write18{bash bin/producedata.sh examples/afh.conf ../data/current.ledger}
%\write18{python degree_tree.py} %run to produce example graph file

\usepackage{fullpage}


\newcommand{\HRule}{\rule{\linewidth}{0.5mm}}
\newcommand{\insertplot}[1]{\includegraphics[width=0.9\linewidth, keepaspectratio=true]{build/#1}\\[1cm]}

\begin{document}

\begin{titlepage}

\begin{center}


% Upper part of the page
\includegraphics[width=0.33\textwidth]{templates/afh/logo.jpg}\\[1cm]

\textsc{\LARGE Bestandsbuch}\\[1.5cm]

\textsc{\Large afh}\\[0.5cm]


% Title
\HRule \\[0.4cm]
{ \huge \bfseries Übersicht meiner Finanzen}\\[0.4cm]

\HRule \\[1.5cm]

\vfill

% Bottom of the page
{\large \today }
{\large \currenttime }
\\[4cm]
{\tiny
  Icons by
  \includegraphics[width=1.5cm]{templates/afh/picol.png}\\
  http://picol.org/
}

\end{center}

\end{titlepage}


\tableofcontents

\chapter*{Einleitung}

``I kept account of every farthing I spent, and my expenses were carefully calculated. 
Every little item, such as omnibus fares or postage or a couple of coppers spent on newspapers, would be entered, and the balance struck every evening before going to bed.
That habit has stayed with me ever since, and I know that as a result, though I have had to handle public funds amounting to lakhs, I have succeeded in exercising strict economy in their disbursement, and instead of outstanding debts have had invariably a surplus balance in respect of all the movements I have led.
Let every youth take a leaf out of my book and make it a point to account for everything that comes into and goes out of his pocket, and like me he is sure to be a gainer in the end.
'' -- M.K.Gandhi autobiography

\chapter{Budget}

\begin{figure}
\caption{Kostenaufschlüsselung des aktuellen Monats}
\insertplot{monthexpensepie}
\end{figure}

\verbatiminput{../data/budget_inc.ledger}

\section{Verlauf des aktuellen Monats}

Negative Werte zeigen verfügbare Mittel an.  Positive Werte weisen auf Verausgabungen hin.

\verbatiminput{build/budget.txt}

\chapter{Aktiva}

\verbatiminput{build/assets.txt}

\chapter{Passiva}

\verbatiminput{build/liabilities.txt}

\chapter{Transaktionen}

\section{Girokonto}

Saldo des letzten Jahres bis dato :

\insertplot{checking-1yearbalance}

Saldo im \monthname :

\insertplot{checking-1monthbalance}

Transaktionen der letzen 7 Tage :

\verbatiminput{build/checking-1trans.txt}

\chapter{Netto Wert}

\begin{itemize}

\item Die Bilanz meiner Aktiva zu meinen Passiva zeigt den Netto Wert: \input{build/networth.txt}
\item Abzüglich langfristiger Investitionen und dem Darlehenskonto ist meine Liquidität \input{build/liquidity.txt}
\item Die Ausgaben mit den Einnahmen zu verrechnen zeigt den Kapitalfluss, oder den Netto-Gewinn, bzw. -Verlust (negativ = Gewinn, positiv = Verlust): \input{build/cashflow.txt}

\end{itemize}

\chapter{Prognose}

\section{Girokonto}

\insertplot{checking-1forecast}

\verbatiminput{build/checking-1forecast.txt}

\end{document}
